\section{Pointer Safety}

One of the primary advantages of C is that it gives the programmer a model that is very close to hardware, allowing access to essentially arbitrary areas of memory.
Although this is very powerful, it can also be very dangerous and numerous bugs have arisen from programmers not putting careful checks on their memory operations.
These bugs can lead to corrupted data, or potentially a security vulnerability.

\subsection{Example: Buffer Overflow Vulnerability}

This first example displays a pointer checking bug that a malicious user can exploit, by allowing user input to overwrite a variable it should not be able to \cite{one1996smashing}.

\begin{minted}[fontsize=\footnotesize,frame=single]{c}
#include <stdio.h>
#include <string.h>

int main(void){
    char user[16];
    char pass[16];
    int userid = 0;

    printf("Username: ");
    gets(user);
    printf("Password: ");
    gets(pass);

    if(PasswordCorrect(user, pass))
        userid = GetUserId(user);

    if(userid == 0){
        printf("Invalid username or password.");
        return;
    }

    ...
}
\end{minted}

\begin{minted}[fontsize=\footnotesize,frame=single]{c}
// In stdio
char *gets(char *buf)
{
    int c;
    char *s;

    for (s = buf; (c = getchar()) != '\n';)
        if (c == EOF)
            if (s == buf)
                return (NULL);
            else
                break;
        else
            *s++ = c;
    *s = 0;
    return (buf);
}
\end{minted}

In this example, the main function simulates a login, the user is asked to enter a username and password, these are checked against each other, the user id is fetched and if \verb!userid! is still 0, they are kicked out as they have failed to log in.
The problem arises from the implementation of \verb!gets! (a simplified version taken from Apple's libc - the original version printed a warning notifying the user that \verb!gets! is unsafe).

The variable \verb!s! is set to point to the first byte of \verb!buf!, and for each character read in through \verb!getchar! that isn't a newline or end-of-file, the memory location pointed to by \verb!s! is set to that character and \verb!s! is incremented.
The problem here is that there is no bounds checking performed on \verb!s!.

The attack occurs as follows:

\begin{enumerate}
\item The attacker provides a password that is longer than 16 characters.
\item \verb!gets! is called, with \verb!pass! being passed in as the argument.
\item The \verb!for! loop continues until a newline in countered, in this case continuing to write past the memory allocated for \verb!pass!.
\item \verb!userid! is positioned in memory after the end of \verb!pass!, so when \verb!gets! writes past the end of \verb!pass!, it overwrites the value of \verb!userid!.
\item \verb!userid!, is originally set to zero, and assumed by the programmer to be unchanged unless the password and username match, has been checked to a non-zero value.
\item The attacker is given a valid \verb!userid!, and can even set the userid to be whatever they want by varying the 17th character of the input password.
\end{enumerate}

\subsection{Example: Heartbleed}

A more topical example is the Heartbleed vulnerability found in the OpenSSL library - an open source implementation of the SSL and TLS protocols \cite{heartbleedBlog}.
Where, in a previous example the attacker can write past the allocated memory to overwrite other values, the Heartbleed bug allows the attacker to read past the memory allocated for the legitimate information and read values stored in RAM (such as passwords and private keys).

The bug is contained in the following code \cite{sslSrc}:

\begin{minted}[fontsize=\footnotesize,frame=single]{c}
/* Read type and payload length first */
hbtype = *p++;
n2s(p, payload);
pl = p;
...
unsigned char *buffer, *bp;
int r;

buffer = OPENSSL_malloc(1 + 2 + payload + padding);
bp = buffer;
...
/* Enter response type, length and copy payload */
*bp++ = TLS1_HB_RESPONSE;
s2n(payload, bp);
memcpy(bp, pl, payload);
\end{minted}

The variable \verb!p! points to a \verb!ssl3_record_st!, the incoming message.
The first byte of the message is read (the type of the record) and is stored in \verb!bhtype!, and then the \verb!n2s! macro takes two bytes from \verb!p! and stores them in the variable \verb!payload!, that is designed to contain the length of the message.
Finally, \verb!pl! is set to contain the rest of the received message.
Note that the received message is received from a potentially untrusted source.

A buffer is created and has \verb!1 + 2 + payload + padding! allocated for it.

The response message is created in \verb!bp! and its first byte (its record type) being set to \verb!TLS1_HB_RESPONSE!, and with the macro \verb!s2n! copying two bytes from \verb!payload! into \verb!bp!.
Finally, \verb!memcpy! is called, that copies \verb!payload! bytes of \verb!pl! into \verb!bp!.
This was originally intended to copy the rest of the heartbeat message back into the response but is the line that introduces the vulnerability.

The bug comes into play when the variable \verb!payload!, originally extracted from the adversary originating message specifies a size bigger than the rest of the package.
\verb!memcpy! will copy past the end of the legitimate data stored in \verb!pl! and start copying bytes into \verb!bp! of other variables and essentially whatever happens to be sitting in RAM nearby \verb!pl! at the time.
Finally \verb!bp! is sent back to the adversary.


\subsection{Example: Returning a Pointer to the Stack}
This final example shows a nasty bug, one that potentially could be used for a security vulnerability, but because the code normally does not function correctly it hasn't been a major exploit \cite{returnLocal}.
This bug stems from the programmers misunderstanding of the lifetime of variables on the stack.

\begin{minted}[fontsize=\footnotesize,frame=single]{c}
Triangle *CreateTriangle(Point *a, Point *b, Point *c){
    Triangle T;
    T.a = a;
	T.b = b;
    T.c = c;
    return &T;
}
\end{minted}

The triangle \verb!T! is created as a local variable, on the stack and modified.
A reference to it is then returned from the function.
However, when the function exits, the local variables lifespan ends, resulting in the returned pointer pointing to deallocated space that will probably be overwritten on the next function call.

\subsection{Spatial and Temporal Pointer Safety}

There are two types of safety that can be provided when dealing with pointers - temporal and spatial.
Temporal safety ensures that the pointer points to a valid area of memory at the time that it is dereferenced, while spatial safety just ensures that the pointer points to a valid area of memory at a certain time.
In the examples provided above, the first two were violations of spatial pointer safety - in the first data was written beyond the end of the object it was meant to write to and in the second, data was read beyond the end of the object it was meant to write to.
The last example was a violation of temporal pointer safety - a pointer is returned an object that is past the end of its lifetime.

The methods explored in this dissertation provide spatial safety, but not complete temporal safety.
Consider the following code:

\begin{minted}[fontsize=\footnotesize,frame=single]{c}
int *a=malloc(sizeof(int));
int *b=a;

free(a);

printf("%d\n", *a);
printf("%d\n", *b);
\end{minted}

The methods implemented in this dissertation will catch the memory access error introduced by the first \verb!printf! statement, but not by the second.
This is because the information about pointer validity is associated with the pointer, and so when \verb!a! is freed the information associated with it indicates that it does not point to valid memory, but the information associated with \verb!b! is not likewise updated.`

Therefore the techniques implemented in this dissertation do not provide complete pointer safety, but provide an additional layer of robustness.

\section{Allowances in the C standard}

The C99 standard defines loose rules for pointer related operations, that allow the transformations performed by Bandage to result in a program that adhers to it \cite{c99all}.

By omission in section \verb!6.2.6.1! of the standard, the representation of the pointer type in unspecified.
This allows pointers to be represented by more informative data structures than just the address they point to, for example a fat pointer representation.

The C standard places the following notable restrictions on pointer types (unless otherwise stated, these are contained in section \verb!6.3.2.3!).

\begin{itemize}
\item A pointer to any object type may be converted to a pointer to void and back again; the result shall compare equal to the original pointer.
\item An integer constant expression with the value 0, or such an expression cast to type \verb!void *!, is called a null pointer constant.  
If a null pointer constant is converted to a pointer type, the resulting pointer, called a null pointer, is guaranteed to compare unequal to a pointer to any object or function.
Conversion of a null pointer to another pointer type yields a null pointer of that type.  
Any two null pointers shall compare equal.
\item An integer may be converted to any pointer type.
Except as previously specified, the result is implementation-defined, might not be correctly aligned, might not point to an entity of the referenced type, and might be a trap representation.
\end{itemize}

These restrictions place \verb!void! pointers as the lowest common denominator, and state that NULL must be equivalent for all pointer types.

However, it also provides a number of allowances that pointer safety schemes can use:

\begin{itemize}
\item The value of a pointer becomes indeterminate when the object it points to (or just past) reaches the end of its lifetime (\verb!6.2.4.2!).
\item When an expression that has integer type is added to or subtracted from a pointer, the result has the type of the pointer operand.
If both the pointer operand and the result point to elements of the same array object, or one past the last element of the array object, the evaluation shall not produce an overflow; otherwise, the behavior is undefined.
If the result points one past the last element of the array object, it shall not be used as the operand of a unary \verb!*! operator that is evaluated. (\verb!6.5.6!)
\item If an invalid value has been assigned to a pointer, the behaviour of the \verb!*! pointer is undefined.
Among the invalid values for dereferencing a pointer by the unary \verb!*! operator are a null pointer, an address inappropriately aligned for the type of object pointed to, and the address of an object after the end of its lifetime. (\verb!6.5.3.2!).
\end{itemize}

The first point would allow creation of a simple temporal pointer safety scheme where the pointer parameter passed to the \verb!free! function is set to null after the function returns, while adhering to the C standard.

Bandage makes use of the latter points, allowing undefined behaviour in the case of pointers pointing past the ends of arrays (or objects) being dereferenced.
However in practice, it is common for programmers to manipulate pointers such that they point out of the bounds of arrays (2 past the end, or before the start), and this is expected to be defined behaviour.
Bandage is conservative and only causes undefined behaviour for dereferences to these pointers.

\section{Working with Pointers in LLVM IR}

During this chapter, types will be referred to in LLVM notation.
An \verb!i32! stands for a 32-bit integer, a \verb!i8! stands for an 8 bit integer (or char) and a pointer to a type is denoted by the type followed by an asterisk.

In LLVM IR, space on the stack is set aside with the \verb!alloca! instruction, that returns a pointer to the allocated memory \cite{llvmLangRef}.
Therefore the line \verb!int a! in C is converted into \verb!%a = alloca i32! in LLVM IR.
It should be noted that the variable \verb!%a! is actually the type of \verb!i32*!.
This means that in the declaration \verb!int *a! will create in LLVM IR the variable \verb!%a! of type \verb!i32**!.

The \verb!store! instruction takes a piece of data and a pointer and stores the data in the area pointed to.
The following C code turns into the following LLVM code:

\begin{minipage}[t]{0.5\linewidth}
\begin{minted}[fontsize=\footnotesize,frame=single]{c}
int a;
int b;

a = b;
\end{minted}
\end{minipage}
\begin{minipage}[t]{0.5\linewidth}
\begin{minted}[fontsize=\footnotesize,frame=single]{llvm}
%a = alloca i32
%b = alloca i32
%1 = load i32* %b
store i32 %1, i32 %a
\end{minted}
\end{minipage}

Here, spaces on the stack for the variables \verb!a! and \verb!b! are created, and pointers to these spaces are stored in the variables \verb!%a! and \verb!%b! respectively.
The value stored in the area pointed to by \verb!%b! is loaded, and then stored in the area pointed to by the variable \verb!%a!.

\begin{minipage}[t]{0.5\linewidth}
\begin{minted}[fontsize=\footnotesize,frame=single]{c}
int *c;
int d;
...


*c = d;
\end{minted}
\end{minipage}
\begin{minipage}[t]{0.5\linewidth}
\begin{minted}[fontsize=\footnotesize,frame=single]{llvm}
%c = alloca i32*
%d = alloca i32
...
%1 = load i32** %c
%2 = load i32* %d
store i32 %2, i32* %c
\end{minted}
\end{minipage}

Here, in the C code we are assigning the value held in \verb!d! to the memory location pointed to by \verb!c! (assume that \verb!c! is set to a valid memory address during the \verb!...! otherwise the code results in undefined behaviour).

In LLVM IR, the variable \verb!%c! is created with type \verb!i32**! and the variable \verb!%d! is created with type \verb!i32*!.
The value stored in the memory pointed to by \verb!%c! is loaded into \verb!%1!, so \verb!%1! now contains the address pointed to by \verb!c! in the C code.
The value stored in the memory pointed to by \verb!%d! is loaded into \verb!%2! (the value contained within \verb!d! in the C code).
Finally the memory pointed to by \verb!%1! is set to the value contained in \verb!%2!.

The last interesting instruction for working with pointers is the GEP instruction.
GEP stands for `Get Element Pointer', and is used for performing pointer arithmetic (it itself does not peform and memory access) \cite{llvmGEP}.

\begin{minipage}[t]{0.3\linewidth}
\begin{minted}[fontsize=\footnotesize,frame=single]{c}
int *a;
int *b;
int c;
int *d;



a = b + 3;




c = d[3];
\end{minted}
\end{minipage}
\begin{minipage}[t]{0.7\linewidth}
\begin{minted}[fontsize=\footnotesize,frame=single]{llvm}
%a = alloca i32*
%b = alloca i32*
%c = alloca i32
%d = alloca i32*

%1 = load i32** %b
%2 = getelementptr %1, i32 0, i32 3
store i32* %2, %a

%3 = load i32** %d
%4 = getelementptr %3, i32 0, i32 3
%5 = load i32* %4
store i32 %5, %c
\end{minted}
\end{minipage}

\verb!%1! contains the value of \verb!b! and the getelementptr is used for the pointer arithmetic.
The variable \verb!%1! is of type \verb!i32*!.
The second parameter of the GEP specifies how many of the size of the type of the first parameter we want to add.
In this case, and in most cases it is zero, saying that we don't want to add any of \verb!sizeof(i32*)! to the first parameter.
The third parameter specifies how many of the size of the type of the value contained within the first parameter we want to add, and in this case, we want to add \verb!3 * sizeof(i32)!.
Therefore the GEP returns:

\begin{verbatim}
%2 = %1 + 0 * sizeof(i32*) + 3 * sizeof(i32)
\end{verbatim}

This address is then stored in the memory pointed to by \verb!%a!.

The second operation can be rewritten as \verb!c = *(d + 3)!, so starts the same as the previous operation, except once the address of \verb!d+3! has been calculated, it is dereferenced to get the value contained there.

