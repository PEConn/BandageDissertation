Since C is such a popular language and pointer based bugs are both widespread and highly exploitable, there have been many efforts to introduce pointer safety to C, and many different approaches.

LLVM's address sanitizer \cite{llvmAddrSan, llvmAddrSanAlgo} can be considered state of the art and  is capable of detecting out-of-bounds accesses on the heap, stack and for globals, use-after-free, and some use-after return bugs.

It does this by creating a copy of memory, called shadow memory, where 1 byte of shadow memory maps to 8 bytes of real memory.
This takes advantage of the fact that \verb!malloc! is guaranteed to return an 8-byte aligned segment of memory, therefore a value of 0 in shadow memory means the corresponding main memory is valid, a negative value means the corresponding main memory is invalid and a positive value of $n$ means the first $n$ bytes are valid and the rest are invalid.

The \verb!malloc! and \verb!free! functions are modified so as to mark the shadowed areas of memory as valid and poisoned respectively.
Additionally, \verb!malloc! is modified so that the memory surrounding that allocated to the program is poisoned to prevent overflows.

However, address sanitizer provides no mapping between the valid areas of memory and variables.
It would be possible for pointer arithmetic to be used to to still cause a buffer overflow into another variables' valid area of memory, though it must jump over the poisoned area.

Hardbound criticizes prior work, saying that it either introduces high overhead, isn't complete or introduces incompatibility \cite{devietti2008hardbound}.
In contrast it proposes to shift the bounds checking to hardware, and akin to SoftBound stores the bounds information separately from the pointer.

\section{The Jones and Kelly System}

An example of the table based approach is the Jones and Kelly approach \cite{jones1997backwards}.
At runtime an ordered list of objects in memory is maintained by tracking the uses of \verb!malloc! and \verb!free!.
It makes use of the fact that every valid pointer-valued expression in C derives its results from exactly one original storage object.
During pointer arithmetic, the referent object (the object pointed to) is identified in the object list using the operand pointer.
The bounds information is retrieved from the object list and used to check if the result is in bounds.

The Jones and Kelly bounds checker uses a strict interpretation of the C standard, where pointers cannot point to invalid memory areas.
Therefore out of bounds pointers are marked as such in a non-recoverable way (they are set equal to \verb!-2!).
To account for the standard allowed practice of generating a pointer pointing one past the end of an array, the bounds checker increases all arrays by 1 element.

In \textit{A Practical Dynamic Buffer Overflow Detector} \cite{ruwase2004practical} it was found that 60\% of programs tested did not adhere to the C-standard assumed in the Jones and Kelly approach and were therefore broken by the tactic of signifying illegal pointers by setting them to \verb!-2!.

To combat this, they created a new approach, where the creation of an out of bounds pointer would result in the creation of an Out Of Bounds object created on the heap which contains the address of the pointer and the referent object originally pointed to.
These Out Of Bounds objects are stored in a hash table.
Therefore on dereference, both the object list and the out of bounds hash table may be consulted to determine the validity of the pointer.
In order to reduce the overhead from these two lookups, only strings are bounds checked on the rationale that they are the tool used in buffer overflow attacks.

\textit{Baggy Bounds Checking} \cite{akritidis2009baggy} is an alternate optimization of the Jones and Kelly system, based on reducing the lookup time.
On a memory allocation, the size of the object is padded to the next power of two, enabling the size of the allocated memory to be stored more compactly as $lg_2(\mbox{size})$ taking the size of a single byte.
Due to the lower memory overhead of a entry, a constant sized array is used instead of an object list.
This allows a quick and constant time address calculation to be performed.
Alternate methods are used for dealing with pointers pointing past the end of arrays as adding one element to an array could double the size it could take up.

This approach does not prevent out of bounds accesses as the size associated with the pointer (the allocated bounds) is larger than the size of the object (the object bounds), so it is still possible to exceed the bounds of the object.
However it prevents dangerous overflows, since a pointer cannot access memory of an object that it was not created for.

\section{SoftBound}

One of the two main systems implemented in this dissertation is SoftBound \cite{nagarakatte2009softbound}.
It is a compile-time transformation that stores information about the valid area of memory associated with a pointer separately from the pointer.

By storing information separately from the pointer, memory layout doesn't change, enabling binary compatibility and reducing implementation effort, however it does require a search for suitable bounds information on pointer dereference.
Additionally the paper contains a proof that spatial integrity is provided by checking the bounds of pointers on a store or load.

